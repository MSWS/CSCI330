\documentclass{article}

\usepackage[tmargin=0.3in,bmargin=0.25in]{geometry}
\usepackage{amsmath, amssymb, amsthm}
\usepackage{listings}
\usepackage{multicol}
\usepackage{enumerate,enumitem}

\title{CSCI 330 Lab 1}
\author{Isaac Boaz}

\begin{document}
\maketitle

\begin{enumerate}
    \item \begin{enumerate}[label=\alph*.]
              \item False. All keys are necessarily a super key.
              \item True.
              \item False. Each \textbf{column} is called an attribute.
              \item False. \textbf{Rows} are known as tuples.
              \item True.
              \item True.
              \item True.
              \item True.
          \end{enumerate}
    \item \begin{tabular}{|c|c|c|}
              \hline
              X & Y & Z   \\
              \hline
              3 & 5 & 100 \\
              \hline
          \end{tabular}
    \item \begin{equation*}
              \sigma_{building=\text{Taylor}}(department) \bowtie \sigma_{salary>=80000}(instructor)
          \end{equation*}
    \item \begin{equation*}
              \sigma_{Category=\text{Electronics}\land Price>500}(Product)
          \end{equation*}
    \item \begin{equation*}
              \Pi_{CustomerName, PhoneNumber}(Customers)
          \end{equation*}
    \item \begin{equation*}
              \Pi_{CustomerName, Address}((\sigma_{Quantity>10}(OrderDetails) \bowtie Orders) \bowtie Customers)
          \end{equation*}
\end{enumerate}

\end{document}