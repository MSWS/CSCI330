\documentclass{article}

\usepackage[letterpaper]{geometry}
\usepackage{listings}
\usepackage{multicol}
\usepackage{multirow}

\title{CSCI 330 Project Report}
\author{Isaac Boaz}

\begin{document}
\maketitle

\section{Introduction}

From the source code, we can observe that the front-end has no knowledge of the back-end implementation. For all it cares, the back-end could be throwing away the data and returning random values.
The benefit in this case is our back-end could be a flat-file system, a mongodb database, an SQL server, or any other data system.
Rather than being tightly coupled with implementation, the front-end interacts with the back-end using a RESTful API, with PUT, GET, DELETE, and POST requests.

\vspace{1em}

\begin{tabular}{|c|c|l|}
    \hline
    \multicolumn{3}{|c|}{\textbf{Task Blueprint}}                \\
    Field    & Type          & Description                       \\
    \hline
    \_id?    & string / null & The id of the task                \\
    text     & string        & The text of the task              \\
    day      & string        & The day the task is due           \\
    reminder & boolean       & Whether or not to remind the user \\
    \hline
\end{tabular}

\begin{itemize}
    \item \textbf{CREATE} - \textit{POST /tasks} Inserts a new \textbf{Task} \\
          \textbf{Returns}
          A JSON object under the \lstinline|result| key with the \textbf{Task} with \textit{\_id} populated. \\
    \item \textbf{READ} - \textit{GET /tasks/(id)} Fetches a specific \textbf{task} or (if unspecified) all \textbf{tasks} \\
          \textbf{Returns}
          A JSON array under the \lstinline|result| key, if ID was specified then the task that matches it will be in the array, if no task matches, it will be empty.
    \item \textbf{UPDATE} - \textit{PUT /tasks/\textless id\textgreater} Toggles the given task's reminder boolean \\
          \textbf{Returns}
          A JSON object under the \lstinline|result| key with the updated \textbf{Task}. \\
    \item \textbf{DELETE} - \textit{DELETE /tasks/\textless id\textgreater} Deletes the given task \\
          \textbf{Returns}
          A JSON object with \lstinline|result| mapped to a boolean.
\end{itemize}

\pagebreak

\section{Implementation}
The workflow for the front-end app is simple.

\begin{itemize}
    \item When the application loads:
          \begin{enumerate}
              \item The application sends a GET request to the back-end for all tasks.
              \item The application generates a list of tasks from the response.
          \end{enumerate}
    \item When a user wishes to make a task:
          \begin{enumerate}
              \item The application gathers the necessary information. (Text, Day, Reminder)
              \item The application sends a POST request to the back-end with the information.
              \item The application maintains an internal list of tasks, and adds the new task to the list.
          \end{enumerate}
    \item When a user double clicks a task:
          \begin{enumerate}
              \item The application sends a PUT request to the back-end with the task's ID.
              \item The application maintains an internal list of tasks, and toggles the reminder boolean of the task.
              \item The application updates the UI to reflect the change.
          \end{enumerate}
    \item When a user wishes to delete a task:
          \begin{enumerate}
              \item The application sends a DELETE request to the back-end with the task's ID.
              \item The application maintains an internal list of tasks, and removes the task from the list.
              \item The application updates the UI to reflect the change.
          \end{enumerate}
\end{itemize}

\section{Differences}
\begin{itemize}
    \item \textbf{Error Handling}: The MongoDB implementation will safely reject invalid requests (null / missing text or malformed data types) whereas the MySQL implementation will internally crash and requires the user to restart the server.
    \item \textbf{Metadata}: MongoDB returns additional metadata about each task, such as the \textit{createdAt}, \textit{updatedAt}, and \textit{\_\_v} fields.
    \item \textbf{Boolean Types}: MongoDB has a boolean type, whereas MySQL does not. MySQL uses an integer type to represent boolean values.
    \item \textbf{Security}: One additional consideration is that each type of backend handles injection / malicious input differently.
    \item \textbf{IDs}: MongoDB returns a 24-character ID per task created, whereas MySQL returns a 36-character UUID.
\end{itemize}

\end{document}